%!TEX root = main.tex
\section{기타 동사}

\subsection{목표 어휘}
% \Entry 부분을 수정하시면 됩니다.
% word: 표준어형
% pred: 예상형태
% feat: 유의점
%       feat에 들어가는 인자는 쉼표(,)로 구분되며,
%       ms, sp, dp 세 종류가 있습니다.
%       feat에 들어간 인자가 볼드체로 표시됩니다.
% desc: 설명
% qstn: 유도질문
% advq: 심화질문
%
% 그리고 줄바꿈이 필요할 땐 엔터가 아니라 '\\'를 사용해주세요(중요!!).

\Entry{
  word={매맞다/매맞아라},
  pred={메맞다/메맞어라},
  feat={sp},
  imag={img/refs/매맞다-매맞아라},
  desc={V1 /ㅐ/ 장단음 \\ V1 /ㅐ-ㅔ/ 대립},
  qstn={어린 시절 부모님은 엄한 분이셨나요? / 어른이 혼을 내면서 회초리 같은 것을 떄리고 있으면, 아이가 무엇을 맞고 있다고 하나요?},
  advq={(매를 맞은 경우) 그때는 친구분들도 대부분 그랬나요? \\ (매를 맞지 않은 경우) 혹시 부모님이 그러시지 않은 이유를 아시나요?}
}

\Entry{
  word={잃다/잃어버리다},
  pred={잃다/잃어베리다, 잃어뻘다, 잃어뿔다, 잃어삐리다},
  feat={ms},
  desc={/잊어버리다/와의 어휘 병합, 의미 차이},
  qstn={소중하게 가지고 계시던 게 없어졌던 적이 있으신가요? / 가지고 있던 물건이 갑자기 어디론가 없어지면 그것을 어떻게 했다고 하시나요?},
  advq={잃어버리셔서 많이 슬프셨겠어요. 도둑맞거나 한 것은 아니지요?}
}

\Entry{
  word={잊다/잊어버리다},
  pred={잊다/잊어베리다, 잊어뻘다, 잊어뿔다, 잊어삐리다},
  feat={ms},
  desc={/잃어버리다/와의 어휘 병합, 의미 차이},
  qstn={생각하고 있던 것이 머릿속에서 생각 나지 않을 때는 어떻게 했다고 하시나요?},
  advq={무언가 잊어버리셔서 곤란하신 적이 있었나요?}
}

\newpage
\Entry{
  word={가르치다},
  pred={갈치다, 겔치다},
  feat={ms},
  imag={img/refs/가르치다},
  desc={/가리키다/와의 어휘 병합, 의미 차이},
  qstn={최근에 새로운 사실을 배운 적 또는 알려주신 적이 있으신가요? 선생님이 학생에게 이런저런 것을 알려주는 것을 무엇을 한다고 하나요?},
  advq={정말 많은 도움이 되었겠어요! 그걸 다른 분에게도 말씀해주신 적이 있을까요?}
}

\Entry{
  word={가리키다},
  pred={갈치다, 겔치다},
  feat={ms},
  imag={img/refs/가리키다},
  desc={/가르치다/와의 어휘 병합, 의미 차이},
  qstn={윗분을 손가락으로 이렇게(허공을 가리키며) 하면 버릇없다고 하잖아요, 어떻게 하면 버릇이 없다고 하는 건가요?},
  advq={그럼 윗분을 예의 바르게 콕 집으려면 어떻게 해야 되나요?}
}

% new lexicon
\Entry{
  word={묻다[問]},
  pred={무꼬, 물꼬, 물코},
  feat={},
  desc={`-고' 결합 시 자음의 발음, /묻다/[埋]와의 성조·음장 차이},
  qstn={%
    음식을 사러 시장에 가셨어요. 그런데 앞에 있는 음식이 마음에 들어서 가격을 알고 싶어요. 그러면 보통 어떻게 하시나요? \\
    다른 곳으로 여행을 나갔는데, 낯선 곳이라 길을 잃었어요. 그런데 마침 옆에 그 동네 사람이 보이네요. 그러면 어떻게 하시겠어요?
  },
  advq={}
}
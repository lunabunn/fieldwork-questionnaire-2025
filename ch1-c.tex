%!TEX root = main.tex
\section{자연과 일상}

\subsection{자유발화 질문}
\begin{itemize}[noitemsep]
  \item 어린 시절에는 주로 무엇을 하고 노셨나요?
  \item 작은 동물들을 잡으며 노신 적이 있나요? 어떤 동물이었나요?
  \item 하루 일과가 어떻게 되세요? 시간대별로 설명해주실 수 있나요?
\end{itemize}

\subsection{문법 유도질문}
※ 시간, 장소, 동반 등의 의미를 가진 조사를 잘 확인하도록 합니다.

\begin{enumerate}[noitemsep]
  \item 주말에 주로 어떤 활동을 하시나요?
  \item 어디 가서 하시나요?
  \item 누구 데리고 가세요?
\end{enumerate}

\subsection{목표 어휘}
% \Entry 부분을 수정하시면 됩니다.
% word: 표준어형
% pred: 예상형태
% feat: 유의점
%       feat에 들어가는 인자는 쉼표(,)로 구분되며,
%       ms, sp, dp 세 종류가 있습니다.
%       feat에 들어간 인자가 볼드체로 표시됩니다.
% desc: 설명
% qstn: 유도질문
% advq: 심화질문
%
% 그리고 줄바꿈이 필요할 땐 엔터가 아니라 '\\'를 사용해주세요(중요!!).

\Entry{
  word={가위바위보},
  pred={-},
  feat={sp},
  imag={img/refs/가위바위보},
  desc={2, 4음절 /ㅟ/},
  qstn={이기고 지는 것을 가르기 위해 이렇게 이렇게 (조사자들끼리 가위바위보를 하는 시늉을 함) 하는 것을 무엇이라고 하나요?},
  advq={실제로 가위바위보를 할 때에 어떻게 말하시나요? (억양 확인)}
}

\Entry{
  word={금(선)/긋다},
  pred={금/기리다},
  feat={ms,sp},
  imag={img/refs/금긋다-금그어라},
  desc={금(금속)과의 성조/장단 차이, \\ 금(선)과 금(갈라짐)의 차이 확인},
  qstn={땅따먹기 놀이를 할 때, (금을 긋는 시늉을 하며) 바닥에 이렇게 하는 걸 뭐라고 하시나요?},
  advq={벽이나 담장이 갈라지면 어떻게 됐다고 하시나요? (금가다)}
}

\Entry{
  word={윷},
  pred={윳ː},
  feat={sp,dp},
  imag={img/refs/윷},
  desc={기저 종성(ㅅ/ㅊ) 확인, \jamoword{zyus/}/늇>윷},
  qstn={명절에 나무토막 네 개를 던지면서 한 칸을 가거나 다섯 칸까지 가거나 하면서 노는 것을 뭐라고 하시나요?},
  advq={
    (`윷놀이'인 경우) 그때 던지는 막대기를 뭐라고 하시나요? (윳ː가락) \\
    이걸 던져서 나오는 눈들은 뭐라고 하시나요? (뙤, 게ː, 걸, 숫, 모) \\
    두 말이 업혀서 같이 가는 것을 부르는 말이 있나요? (둑, 둑동)
  }
}

\Entry{
  word={썰매},
  pred={썰메},
  feat={},
  imag={img/refs/썰매},
  desc={},
  qstn={얼음판이나 눈 위에서 타고 노는 것을 뭐라고 부르시나요?},
  advq={}
}

\Entry{
  word={얼레},
  pred={자세},
  feat={ms},
  imag={img/refs/얼레},
  desc={연줄/낚싯줄 감는 것 명칭 차이 여부},
  qstn={(*사진 자료 참고) 연을 날릴 때 실을 어디에 감으시나요?},
  advq={연줄 감는 거랑 낚싯줄 감는 것을 똑같이 얼레라고 하나요?}
}

\newpage
% new lexicon
\Entry{
  word={봄},
  pred={봄},
  feat={},
  imag={img/refs/봄},
  desc={},
  qstn={(*사진 자료 참고) 꽃 피고 새싹이 돋아나는 따듯한 계절을 뭐라고 하시나요? \\ 사계절을 각각 뭐라고 하시는지 말씀해 주세요.},
  advq={봄에는 뭘 하면서 지내나요? (A. 농사와 식물) \\ 봄에 어떤 음식을 먹나요?}
}

% new lexicon
\Entry{
  word={여름},
  pred={여림},
  feat={},
  imag={img/refs/여름},
  desc={},
  qstn={%
    (*사진 자료 참고) 매미가 울고 장맛비가 내리는 습한 계절을 뭐라고 하시나요? \\
    봄 다음 계절을 뭐라고 하나요?
  },
  advq={여름에는 뭘 하면서 지내나요? 여름에 어떤 음식을 먹나요?}
}

% new lexicon
\Entry{
  word={가을},
  pred={가실},
  feat={sp,dp},
  imag={img/refs/가을},
  desc={\jamoword{ga/z@r}>가을},
  qstn={%
    (*사진 자료 참고) 꽃이 지고 낙엽이 떨어지는 쌀쌀한 계절을 뭐라고 하시나요? \\
    여름 다음 계절을 뭐라고 하나요?
  },
  advq={가을에는 뭘 하면서 지내나요? `가실하다'는 무슨 뜻인가요? (추수하다)}
}

\newpage
\Entry{
  word={겨울},
  pred={저올, 삼동},
  feat={sp,dp},
  imag={img/refs/겨울},
  desc={1음절 /겨/ 구개음화, \jamoword{gye/zvrh}/\jamoword{gye/vrh}>겨울},
  qstn={
    (*사진 자료 참고) 강이 얼고 나무에 잎이 지는 계절을 뭐라고 하시나요? \\
    가을 다음 계절을 뭐라고 하나요? (C13.눈 점화효과 주의)
  },
  advq={겨울에는 뭘 하면서 지내나요? 겨울에 어떤 음식을 먹나요?}
}

\Entry{
  word={춥다},
  pred={춥다, 추와서, 추워서},
  feat={ms},
  desc={/차다/와의 의미 차이},
  qstn={겨울(C09)에는 날씨가 어떻다고 이야기하나요?},
  advq={너무 추우면 무엇이 심하다고 하나요? (추위)}
}

\Entry{
  word={차다, 차갑다},
  pred={차다, 찹다, 차와서},
  feat={ms},
  desc={/춥다/와의 의미 차이},
  qstn={얼음이 어떻다고 이야기하나요? (얼음을 만져 손이 시린 시늉을 하며) \\ 겨울철에는 물이 얼음장같이 어떠하다고 하죠?},
  advq={날이/날씨가 차다고 하기도 하나요? 사람 성격이 차다고 하기도 하나요?}
}

\Entry{
  word={고드름},
  pred={고드름, 고도름},
  feat={},
  imag={img/refs/고드름},
  desc={},
  qstn={겨울에 처마 밑에 주렁주렁 매달리는 얼음덩어리는 뭐라고 하시나요?},
  advq={이게 생기면 어떻게 하시나요? 없앨 때는 어떻게 없애나요?}
}

\newpage
\Entry{
  word={눈(날씨)},
  pred={눈ː},
  feat={sp},
  imag={img/refs/눈(날씨)},
  desc={눈(신체)과의 성조, 장단 차이},
  qstn={추운(C10) 날 비 말고 하늘에서 내리는, 하얗게 쌓이고 잘 뭉치는 것은 무엇인가요?},
  advq={눈이 내리면 뭘 하고 노나요? (C04 썰매 등)}
}

\Entry{
  word={소나기},
  pred={쏘네기, 소네기},
  feat={},
  imag={img/refs/소나기},
  desc={},
  qstn={하늘이 맑다가 갑자기 비가 오면 뭐라고 하시나요? \\ 주로 여름(C07)에 별안간 세차게 쏟아지다가 곧 그치는 비를 뭐라고 하시나요?},
  advq={집 밖에 있는데 소나기가 오면 어떻게 하세요? \\ 소나기를 맞으면 어떻게 옷이랑 몸을 말리나요?}
}

\Entry{
  word={번개, 벼락},
  pred={번게, 버네, 베락},
  feat={},
  imag={img/refs/벼락},
  desc={},
  qstn={
    주로 비가 내릴 때 하늘에서 빛이 번쩍하고 큰 소리가 나는 것을 뭐라고 하시나요? \\
    번개가 나무나 사람에 떨어지면 무엇이 떨어진다고 하시나요? 갑자기 부자가 된 사람을 [이것] 부자라고도 해요.
  },
  advq={집 밖에 있는데 벼락이 치면 어떻게 하세요?}
}

\newpage
\Entry{
  word={새벽},
  pred={세북, 세복},
  feat={},
  desc={},
  qstn={하루 중에서 해가 채 뜨지 않았을 무렵을 뭐라고 하시나요? \\ 하루 중에서 아침이 밝기 직전일 때를 뭐라고 하시나요?},
  advq={보통 몇 시까지를 새벽이라고 하세요? \\ 새벽에 잠이 깬 적 있으세요? 새벽에 깨면 몸이/기분이 어떠세요?}
}

% new lexicon
\Entry{
  word={아침},
  pred={아적, 아칙, 아침},
  feat={},
  desc={},
  qstn={하루 중에 해가 막 뜰 때를 뭐라고 하시나요?},
  advq={보통 몇 시까지를 아침이라고 하세요? 아침 몇 시에 밥을 드세요?}
}

% new lexicon
\Entry{
  word={낮},
  pred={낫ː, 낫ː이, 낫ː을, 낮ː에, 낮ː이로},
  feat={sp},
  desc={기저 종성(ㅅ/ㅈ) 확인},
  qstn={하루 중에 해가 머리 위에 한참 떠 있을 때를 뭐라고 하시나요?},
  advq={낮에는 주로 무엇을 하시나요?}
}

\Entry{
  word={저녁},
  pred={저녁, 저역},
  feat={},
  desc={},
  qstn={하루 중에서 해가 막 질 무렵을 뭐라고 하시나요?},
  advq={보통 몇 시부터를 저녁이라고 하세요? 저녁 몇 시에 밥을 드세요?}
}

\newpage
\Entry{
  word={밤(夜)},
  pred={밤},
  feat={sp},
  desc={밤(음식)과의 성조, 장단 차이},
  qstn={하루 중에 해가 져서 캄캄할 때를 뭐라고 하시나요?},
  advq={보통 몇 시부터를 밤이라고 하세요? 보통 밤 몇 시에 주무시나요?}
}

\Entry{
  word={길다},
  pred={질:다},
  feat={sp},
  imag={img/refs/길다},
  desc={1음절 /기/ 구개음화},
  qstn={(임의로 길이가 다른 두 물체를 제시하며) 이건 이것보다 어때요?},
  advq={}
}

\Entry{
  word={짧다},
  pred={잘럽다, 잘러와서},
  feat={},
  imag={img/refs/짧다},
  desc={},
  qstn={(임의로 길이가 다른 두 물체를 제시하며) 이건 이것보다 어때요?},
  advq={}
}

\newpage
\Entry{
  word={하루, 이틀, \dots 열흘},
  pred={하리, 이틀, 살ː/사을, 날ː/나을, 닷세, 엿세/얏세, 이레, 야드레, 아으레, 열을},
  feat={dp},
  desc={\jamoword{/h@/r@/>/h@/ro/}>하루, 렷새>엿새, \\ 니레/닐헤>이레, 알흐래>아흐레},
  qstn={사흘, 나흘 이런 식으로, 1부터 10까지 날을 세 주세요. \\ (또는 C40.벌레에서 하루살이를 먼저 검출했다면 유도질문에 사용할 수 있다)},
  advq={%
    ``여행을 갔다 오는데 하루가 걸린다''는 문장에, `하루'의 자리에 다른 날짜를 넣어서 천천히 한 번 더 말씀해 주시겠어요? \\
    하루만 사는 벌레를 뭐라고 하시나요? (-살이) \\
    두이레, 세이레라는 말도 쓰시나요?
  }
}

\Entry{
  word={내일},
  pred={넬ː},
  feat={},
  desc={},
  qstn={오늘의 다음날을 뭐라고 하시나요?},
  advq={혹시 내일 말고 다른 말도 있나요?}
}

\Entry{
  word={모레},
  pred={모레, 모리},
  feat={},
  desc={},
  qstn={내일(C24)의 다음날은 무엇이라고 하시나요?},
  advq={혹시 모레 말고 다른 말도 있나요?}
}

\newpage
\Entry{
  word={글피},
  pred={고페},
  feat={dp},
  desc={글픠>글피},
  qstn={모레(C25)의 다음날은 무엇이라고 하시나요?},
  advq={%
    자주 쓰시는 말인가요? \\
    혹시 글피 말고 다른 말도 있나요? 글피의 다음날은 뭐라고 하시나요? (그고페)
  }
}

\Entry{
  word={어제},
  pred={어저께, 어지께},
  feat={ms},
  desc={어제/어저께 의미 차이},
  qstn={오늘의 전날은 무엇이라고 하시나요?},
  advq={혹시 어제 말고 다른 말도 있나요? \\ 어제/어저께라는 말도 쓰시나요? 같은 뜻인가요?}
}

\Entry{
  word={그저께},
  pred={그저께},
  feat={ms,dp},
  desc={`엊그저께'와의 의미 차이 \\ 그젓긔>그저께},
  qstn={어제(C27)의 전날은 무엇이라고 하시나요?},
  advq={혹시 그저께 말고 다른 말도 있나요? \\ 엊그저께라는 말도 쓰시나요? 같은 뜻인가요?}
}

\Entry{
  word={그끄저께},
  pred={그그저께, 그그러께},
  feat={},
  desc={},
  qstn={그저께(C28)의 전날은 무엇이라고 하시나요?},
  advq={혹시 그끄저께 말고 다른 말도 있나요?}
}

\newpage
\Entry{
  word={지렁이},
  pred={거ː시렝이},
  feat={ms,dp},
  imag={img/refs/지렁이},
  desc={\jamoword{gez/que/zi/}/것위>거위(회충),  \\ 디롱이/디룡이>지렁이 \\ 정확한 의미역 파악 필요},
  qstn={비가 오면 땅바닥에 나와 기어다니는 것을 무엇이라고 부르나요?},
  advq={혹시 회충이라는 말도 쓰시나요? 같은 뜻인가요?}
}

\Entry{
  word={미꾸라지},
  pred={미꾸라지, 웅구럭지},
  feat={},
  imag={img/refs/미꾸라지},
  desc={},
  qstn={길고 미끌미끌하고, 추어탕을 끓여 먹는 물고기는 무엇이라고 하시나요?},
  advq={미꾸라지/웅구럭지라는 말은 안 쓰나요? 같은 말인가요? \\ 미꾸라지는 어떻게 잡나요?}
}

\Entry{
  word={개구리},
  pred={게우레기, 게고락지, 머구리},
  feat={},
  imag={img/refs/개구리},
  desc={},
  qstn={풀밭으로 나오기도 하고 물속에 들어가 살기도 하고, 폴짝폴짝 뛰어다니면서 우는 동물을 뭐라고 하시나요?},
  advq={%
    머구리라는 말은 안 쓰나요? 같은 말인가요? \\
    뭔가를 시킬 때마다 반대로만 하는 사람을 뭐라고 하나요? (`청개구리')
  }
}

\newpage
\Entry{
  word={올챙이},
  pred={올쳉이, 복젱이},
  feat={},
  imag={img/refs/올챙이},
  desc={},
  qstn={(*사진 자료 참고) 개구리(C32)가 되기 전에 알에서 막 태어난 것을 뭐라고 하시나요? (C39 새끼 점화효과 주의)},
  advq={올챙이는 어디에 가야 많이 볼 수 있나요?}
}

\Entry{
  word={두꺼비},
  pred={뚜께비},
  feat={},
  imag={img/refs/두꺼비},
  desc={},
  qstn={개구리(C32)랑 비슷한데 조금 크고, 몸이 우둘투둘한 것은 무엇이라고 하시나요?},
  advq={개구리(C32)랑 두꺼비랑 또 어떤 차이가 있나요? \\ 두꺼비집 노래가 있지요? 어떻게 부르나요?}
}

\Entry{
  word={거머리},
  pred={검ː저리},
  feat={},
  imag={img/refs/거머리},
  desc={},
  qstn={논에서 일할 때 다리에 달라붙어서 피를 빨아먹는 것을 무엇이라고 하시나요?},
  advq={거머리가 다리에 달라붙으면/거머리한테 물리면 어떻게 하나요?}
}

\Entry{
  word={달팽이},
  pred={달팡이},
  feat={},
  imag={img/refs/달팽이},
  desc={},
  qstn={등에 동글동글한 집을 달고 더듬이를 뻗고 기어다니는 동물은 뭐라고 하시나요?},
  advq={논밭에 달팽이가 보일 때도 있나요? \\ 달팽이가 기는 속도를 어떻다고 하세요? (니리다, 느리다)}
}

\newpage
\Entry{
  word={다슬기},
  pred={데사리, 민물꼬동},
  feat={ms},
  imag={img/refs/다슬기},
  desc={민물/갯다슬기 어휘 차이 확인},
  qstn={%
    달팽이(C36)이랑 비슷한데 냇물이나 갯벌(D12)에서 사는 것은 무엇이라고 하시나요? 된장국에 넣어먹기도 하는 것이요.
  },
  advq={%
    민물에서 사는 다슬기와 갯벌(D12)에서 사는 다슬기를 다르게 부르나요? (`갯다슬기' 고동, 다시리)
    다슬기는 보통 어떤 식으로/어떻게 잡나요?
  }
}

\Entry{
  word={우렁이},
  pred={우렁},
  feat={},
  imag={img/refs/우렁이},
  desc={},
  qstn={다슬기(C37)랑 비슷한데, 논에서 사는 것은 무엇이라고 하시나요? \\ 조금 더 크고 둥글고 민물에 살고, 쌈밥이나 된장국으로 해먹기도 하고요.},
  advq={우렁이는 보통 어떤 식으로/어떻게 잡나요? \\ 우렁각시 이야기를 아시나요?}
}

\Entry{
  word={새끼(동물)},
  pred={세끼},
  feat={sp},
  desc={새끼(줄)와의 성조, 장단 차이 \\ 1음절 /ㅐ-ㅔ/ 대립},
  qstn={갓 태어난 어린 동물들을 뭐라고 하시나요?},
  advq={소나 돼지가 새끼를 낳으면 무엇을 해줘야 하나요? 사람한테도 새끼라고 하나요?}
}

\newpage
\Entry{
  word={벌레},
  pred={벌거지, 벌게},
  feat={},
  desc={},
  qstn={아주 작고 꼼지락거리며 기어다니거나 날아다니는 것을 뭐라고 하시나요?},
  advq={날아다니는 것, 기어다니는 것, 딱딱한 것 모두 벌레인가요? \\ 주변에서 많이 볼 수 있는 벌레는 무엇이 있나요?}
}

\Entry{
  word={서캐},
  pred={쎄, 서까리},
  feat={},
  desc={},
  qstn={머리에 생기는 이의 알은 무엇이라고 하시나요?},
  advq={서캐가 있으면 어떻게 없애나요?}
}

\Entry{
  word={파리},
  pred={포리},
  feat={},
  imag={img/refs/파리},
  desc={},
  qstn={음식에 날아 앉는 벌레(C40)인데 손을 이렇게 (파리 흉내를 낸다) 비비는 것을 무엇이라고 하시나요?},
  advq={이걸 잡는 데 쓰는 도구는 무엇이라고 하시나요? (포리체)}
}

\Entry{
  word={구더기},
  pred={구데기, 구더리},
  feat={},
  desc={},
  qstn={파리(C42)가 알을 낳으면 태어나는 하얀 벌레(C40)를 뭐라고 하세요? `이것이 무서워서 장 못 담근다'는 속담이 있죠.},
  advq={구더기는 어디에 많이 나오나요? 집 안에 이것이 생기면 어떻게 하세요?}
}

\newpage
\Entry{
  word={벼룩},
  pred={베룩},
  feat={},
  imag={img/refs/벼룩},
  desc={},
  qstn={아주 조그마한데, 톡톡 튀어 다녀서 잡기가 어려운 벌레(C40)는 무엇이라고 하시나요? `뛰어야 이것이지'라는 말이 있죠},
  advq={}
}

\Entry{
  word={벌},
  pred={벌},
  feat={},
  imag={img/refs/벌},
  desc={},
  qstn={집을 짓고 꿀을 만드는 곤충을 뭐라고 하시나요? 집을 건드리면 떼 지어 쫓아와서 공격하는 것이요.},
  advq={%
    벌을 집 안이나 바깥에서 만나면 어떻게 하시나요? \\
    벌침에 찔리는 것을 뭐라고 하나요? (쎄ː다, 쎄이다) 벌에 쏘였을 때 어떻게 하나요?
  }
}

\Entry{
  word={쥐},
  pred={지, 쥐(이중모음)},
  feat={sp},
  imag={img/refs/쥐},
  desc={1음절 /ㅟ/},
  qstn={밤에 찍찍거리고 돌아다니면서 곡식을 훔쳐 먹는 동물을 무엇이라고 하시나요?},
  advq={쥐가 집안에 들어온 적이 있나요? 쥐가 들어오면 어떻게 잡으세요?}
}

\newpage
\Entry{
  word={돼지},
  pred={뒤아지, 돗},
  feat={},
  imag={img/refs/돼지},
  desc={},
  qstn={뚱뚱하고, 꿀꿀거리면서 울고, 고기로 많이 해먹는 동물을 무엇이라고 하시나요?},
  advq={%
    돼지를 부를 때 무슨 소리를 내시나요? 돼지고기는 어떤 부위를 가장 좋아하세요? \\
    돼지는 어디서 키우나요? 돼지들은 주로 뭘 먹이면서 키우나요?
  }
}

\Entry{
  word={고양이},
  pred={괴, 고옝이},
  feat={},
  imag={img/refs/고양이},
  desc={},
  qstn={쥐(C46)을 잘 잡고, 야옹 하고 우는 동물을 뭐라고 하시나요?},
  advq={고양이를 키우시나요? \\ 아시는 분들 중에 고양이를 키우는 분이 계신가요?}
}

\Entry{
  word={개},
  pred={게ː},
  feat={sp, dp},
  imag={img/refs/개},
  desc={1음절 /ㅐ-ㅔ/ 대립, 가히/갛>개},
  qstn={집에서 기르는 동물인데, `멍멍' 하고 짖는 것을 뭐라고 하시나요?},
  advq={}
}

% new lexicon
\Entry{
  word={여우},
  pred={여시, 여꾸},
  feat={ms,dp},
  imag={img/refs/여우},
  desc={다양한 조사 결합형을 확인할 것 \\ /\jamoword{ye/zu/}{\textasciitilde}\jamoword{yez/q/}/ > /여우/},
  qstn={개와 비슷하게 생긴 야생 동물인데, 털빛이 붉고 귀가 뾰족한 것을 뭐라고 하시나요?},
  advq={여우와 관련된 전설을 아는 게 있으신가요? (벡여시)}
}

\newpage
\Entry{
  word={꿩},
  pred={꿩, 꽁},
  feat={sp},
  imag={img/refs/꿩},
  desc={1음절 /ㅝ/},
  qstn={사냥꾼이 많이 잡는 새인데, 숨을 때 머리만 처박는 것을 무엇이라고 하시나요? \\ `이것 대신 닭'이라는 말이 있죠.},
  advq={%
    수놈과 암놈을 따로 부르는 말이 있나요? (장꿩, 젱끼, 숫꽁, 숫딱꿩 / 암꿩, 암탁꿩) \\
    새끼를 따로 부르는 말도 있나요? (가터리)
  }
}

\Entry{
  word={매},
  pred={메ː},
  feat={sp},
  imag={img/refs/매},
  desc={1음절 /ㅔ/ 장단음 및 /ㅐ-ㅔ/ 대립},
  qstn={독수리랑 비슷한 새인데, 길들여서 사냥에 쓰기도 하는 새를 무엇이라고 하시나요? (`이것의 눈'이라는 말이 있죠)},
  advq={}
}

\Entry{
  word={말(동물)},
  pred={말},
  feat={sp},
  imag={img/refs/말(동물)},
  desc={/말/(언어)과의 성조 음장 차이},
  qstn={아주 빠르게 달리고, 마차를 끄는 동물을 뭐라고 하시나요?},
  advq={말은 어디서 키우나요? (C54 마구간)}
}

\Entry{
  word={마구간},
  pred={마ː청, 마ː판, 마ː방, 마ː구},
  feat={},
  imag={img/refs/마구간},
  desc={},
  qstn={말(C53)이 사는 집은 뭐라고 하시나요?},
  advq={}
}

\newpage
% new lexicon
\Entry{
  word={소},
  pred={소},
  feat={},
  imag={img/refs/소},
  desc={},
  qstn={농사할 때 쓰고, 우유가 나오는 동물을 뭐라고 하시나요?},
  advq={소는 어디서 키우나요? (C56 외양간)}
}

\Entry{
  word={외양간},
  pred={마ː구, 소양깐, 오양깐},
  feat={ms, dp},
  imag={img/refs/외양간},
  desc={마구간(C54)과 합류 일어났을 가능성 \\ 오희양/외향>외양간},
  qstn={소(C55)가 사는 집은 무엇이라고 하시나요? `소 잃고 이거 고친다'는 말이 있죠.},
  advq={(`외양간' 계열 어형이 등장하지 않는 경우) 외양간이라는 말은 안 쓰시나요?}
}
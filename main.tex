% !TEX root = main.tex
\documentclass{snu-fl-questionnaire}

% Hangul font setup
\setmainhangulfont{KoPubWorldBatang_Pro}[
  Scale = MatchUppercase,
  UprightFont={* Light},
  BoldFont={* Bold},
  AutoFakeSlant = 0.15
]
\setsanshangulfont{KoPubWorldDotum_Pro}[
  Scale = MatchUppercase,
  BoldFont={* Bold},
]

% If there are problems in Hangul Font setup code,
% run the following line to update your cache:
%   $ fc-cache -fv
%
% Note that on Windows, by default, the user-specific font
% directory (i.e. %LOCALAPPDATA%\Microsoft\Windows\Fonts\)
% is not picked up by fc-cache. Either add this directory
% manually or install the font for all users (C:\Windows\Fonts)
% as administrator using the right-click menu.
%
% Or, you can load fonts by their filenames, e.g.,
%
%\setmainhangulfont{KoPubWorldBatang_Pro}[
%  Scale = MatchUppercase,
%  UprightFont = {KoPubWorld Batang_Pro Light.otf}, % Exact filename
%  BoldFont = {KoPubWorld Batang_Pro Bold.otf},     % Exact filename
%  AutoFakeSlant = 0.15
%]
%\setsanshangulfont{KoPubWorldDotum_Pro}[
%  Scale = MatchUppercase,
%  UprightFont = {KoPubWorld Dotum_Pro Medium.otf}, % Exact filename (or Light)
%  BoldFont = {KoPubWorld Dotum_Pro Bold.otf}       % Exact filename
%]

% Metadata
\title{2025 한국 언어조사 질문지}
\author{서울대학교 인문대학 언어학과}
\date{2025년 10월 31일 \textasciitilde{} 11월 1일}
\printdate{2025년 10월 31일}
\issuedate{2025년 10월 31일}
\tel{(02) 880-6163, 6164}


\begin{document}

\frontmatter
\maketitle
\tableofcontents

\chapter{2025학년도 언어조사 개요}
일정: 2025년 10월 31일(금요일) {\textasciitilde} 11월 1일(토요일)\\
출발: 2025년 10월 31일(금요일) 08시 58분, 광명역\\
숙소: 순천 더소풍펜션\\
\\
숙소까지의 교통편: KTX 및 렌트카\\
\\
조사장소: 전라남도 보성군 벌교읍 보성군노인복지관 일대

\chapter{2025학년도 언어조사 일정}
\begin{tblr}{
  width = \linewidth,
  hlines,
  vlines,
  cells = {c},
  colspec = {c c X[c] c},
  row{1} = {lightgray},
  cell{2}{1} = {r=8}{},
  cell{10}{1} = {r=7}{},
  cell{8}{4} = {r=3}{},
}
\textbf{일자} & \textbf{일정} & \textbf{내용} & \textbf{장소} \\
`25.10.31. & 08:58-11:23 & 이동 (광명역 $\rightarrow$ 순천역) & KTX \\
& 11:30-12:30 & 점심 식사 & 별량시골밥상 \\
& 12:30-13:30 & 조사 장소 이동, 조사 준비 & 렌트카 \\
& 13:30-17:00 & \underline{언어조사} & \underline{보성군노인복지관} \\
& 17:00-18:00 & 저녁 식사 & 명품갈비탕 \\
& 18:00-18:30 & 장소 이동 & 렌트카 \\
& 18:30-21:00 & 조사 내용 회의 & 더소풍펜션 \\
& 21:00- & 휴식 & \\
`25.11.01 & -09:00 & 기상 및 아침 식사 & \\
& 09:00-10:00 & 조사 장소 이동, 조사 준비 & 렌트카 \\
& 10:00-12:30 & \underline{언어조사} & \underline{보성군노인복지관} \\
& 12:30-13:30 & 장소 이동, 점심 식사 & 회정국밥 \\
& 12:30-15:18 & 지역 문화 탐방 & 순천만국가정원 \\
& 15:18-17:43 & 이동 (순천역 $\rightarrow$ 광명역) & KTX \\
& 17:43- & 광명역 도착, 짐 정리 & 서울대
\end{tblr}

\chapter{자료제공인 조사표 (1)}
\Consultant

\chapter{자료제공인 조사표 (2)}
\Consultant

\chapter{자료제공인 조사표 (3)}
\Consultant

\chapter{자료제공인 조사표 (4)}
\Consultant

\chapter{일러두기}
일러둘 내용을 적습니다.


\mainmatter
\chapter{어휘·음운편}
\subsection{일러두기}
일러둘 내용을 적습니다.

% A. 농사와 식물
\include{ch2-a}

% B. 가정과 생활
\include{ch2-b}

% C. 자연과 일상
\include{ch2-c}

% D. 바다
\include{ch2-d}

% E. 사회적 지식
\include{ch2-e}

% F. 신체
\include{ch2-f}

% G. 기타 동사
%!TEX root = main.tex
\section{기타 동사}
\subsection{일러두기}
일러둘 내용을 적습니다.

\subsection{자유발화 질문}
자유발화 질문을 적습니다.

\subsection{목표 어휘}
\Entry{
  word={매맞다/매맞아라},
  pred={},
  feat={ms,sp,dp},
  desc={V1 /ㅐ/ 장단음 \\ V1 /ㅐ-ㅔ/ 대립},
  qstn={어린 시절 부모님은 엄한 분이셨나요? / 어른이 혼을 내면서 회초리 같은 것을 떄리고 있으면, 아이가 무엇을 맞고 있다고 하나요?},
  advq={매를 맞은 경우) 그때는 친구분들도 대부분 그랬나요? \\ 매를 맞지 않은 경우) 혹시 부모님이 그러시지 않은 이유를 아시나요?}
}

\Entry{
  word={잃다/잃어버리다},
  pred={잃다(잃어뻐리다). 일치도, 이른, 이러따, 이러버려따},
  feat={ms,sp,dp},
  desc={잊어버리다와의 어휘 병합, 의미 차이},
  qstn={소중하게 가지고 계시던 게 없어졌던 적이 있으신가요? / 가지고 있던 물건이 갑자기 어디론가 없어지면 그것을 어떻게 했다고 하시나요?},
  advq={잃어버리셔서 많이 슬프셨겠어요. 도둑맞거나 한 것은 아니지요?}
}

\Entry{
  word={잊다/잊어버리다},
  pred={잊어뻐리다. 이저따},
  feat={ms,sp,dp},
  desc={잃어버리다와의 어휘 병합, 의미 차이},
  qstn={생각하고 있던 것이 머릿속에서 생각나지 않을 때는 어떻게 했다고 하시나요?},
  advq={무언가 잊어버리셔서 곤란하신 적이 있었나요?}
}

\Entry{
  word={가르치다},
  pred={가리치다. 가르키는, 가르처},
  feat={ms,sp,dp},
  desc={가리키다와의 어휘 병합, 의미 차이},
  qstn={최근에 새로운 사실을 배운 적 또는 알려주신 적이 있으신가요? 선생님이 학생에게 이런저런 것을 알려주는 것을 무엇을 한다고 하나요?},
  advq={정말 많은 도움이 되었겠어요! 그걸 다른 분에게도 말씀해주신 적이 있을까요?}
}

\Entry{
  word={가리키다},
  pred={가리치다. 가리켜, 일러줘야, 가리키면, 가르처},
  feat={ms,sp,dp},
  desc={가르치다와의 어휘 병합, 의미 차이},
  qstn={윗분을 손가락으로 이렇게(허공을 가리키며) 하면 버릇없다고 하잖아요, 어떻게 하면 버릇이 없다고 하는 건가요?},
  advq={그럼 윗분을 예의 바르게 콕 집으려면 어떻게 해야 되나요?}
}

% H. 친족어
%!TEX root = main.tex
\section{친족어}
\subsection{일러두기}
일러둘 내용을 적습니다.

\subsection{자유발화 질문}
자유발화 질문을 적습니다.

\subsection{목표 어휘}
\Entry{
  word={모친},
  pred={어머니},
  feat={ms,sp,dp},
  desc={},
  qstn={},
  advq={}
}

\Entry{
  word={모친의 모친},
  pred={할메, 할미},
  feat={ms,sp,dp},
  desc={큰마니, 클마니, 클만, 할만 등},
  qstn={},
  advq={}
}

\Entry{
  word={모친의 부친},
  pred={할아버지},
  feat={ms,sp,dp},
  desc={},
  qstn={},
  advq={}
}

\Entry{
  word={모친의 손위 여성형제},
  pred={이모},
  feat={ms,sp,dp},
  desc={},
  qstn={},
  advq={}
}

\Entry{
  word={모친의 손위 여성형제의 남편},
  pred={},
  feat={ms,sp,dp},
  desc={},
  qstn={},
  advq={}
}

\Entry{
  word={모친의 손위 남성형제},
  pred={오삼춘},
  feat={ms,sp,dp},
  desc={V1 /ㅚ/ (외삼촌)},
  qstn={},
  advq={}
}

\Entry{
  word={모친의 손위 남성형제의 아내},
  pred={},
  feat={ms,sp,dp},
  desc={},
  qstn={},
  advq={}
}

\Entry{
  word={모친의 손아래 여성형제},
  pred={이모},
  feat={ms,sp,dp},
  desc={},
  qstn={},
  advq={}
}

\Entry{
  word={모친의 손아래 여성형제의 남편},
  pred={},
  feat={ms,sp,dp},
  desc={},
  qstn={},
  advq={}
}

\Entry{
  word={모친의 손아래 남성형제},
  pred={오삼춘},
  feat={ms,sp,dp},
  desc={V1 /ㅚ/ (외삼촌)},
  qstn={},
  advq={}
}

\Entry{
  word={모친의 손아래 남성형제의 아내},
  pred={},
  feat={ms,sp,dp},
  desc={},
  qstn={},
  advq={}
}

\Entry{
  word={부친},
  pred={아부지},
  feat={ms,sp,dp},
  desc={},
  qstn={},
  advq={}
}

\Entry{
  word={부친의 모친},
  pred={할메, 할미},
  feat={ms,sp,dp},
  desc={큰마니, 클마니, 클만, 할만 등},
  qstn={},
  advq={}
}

\Entry{
  word={부친의 부친},
  pred={할아버지},
  feat={ms,sp,dp},
  desc={},
  qstn={},
  advq={}
}

\Entry{
  word={부친의 손위 여성형제},
  pred={고모},
  feat={ms,sp,dp},
  desc={},
  qstn={},
  advq={}
}

\Entry{
  word={부친의 손위 여성형제의 남편},
  pred={},
  feat={ms,sp,dp},
  desc={},
  qstn={},
  advq={}
}

\Entry{
  word={부친의 손위 남성형제},
  pred={큰아버지},
  feat={ms,sp,dp},
  desc={},
  qstn={},
  advq={}
}

\Entry{
  word={부친의 손위 남성형제의 아내},
  pred={},
  feat={ms,sp,dp},
  desc={},
  qstn={},
  advq={}
}

\Entry{
  word={부친의 손아래 여성형제},
  pred={고모},
  feat={ms,sp,dp},
  desc={},
  qstn={},
  advq={}
}

\Entry{
  word={부친의 손아래 여성형제의 남편},
  pred={},
  feat={ms,sp,dp},
  desc={},
  qstn={},
  advq={}
}

\Entry{
  word={부친의 손아래 남성형제},
  pred={},
  feat={ms,sp,dp},
  desc={},
  qstn={},
  advq={}
}

\Entry{
  word={부친의 손아래 남성형제의 아내},
  pred={},
  feat={ms,sp,dp},
  desc={},
  qstn={},
  advq={}
}

\Entry{
  word={손위 여성형제},
  pred={},
  feat={ms,sp,dp},
  desc={},
  qstn={},
  advq={}
}

\Entry{
  word={손위 여성형제의 남편},
  pred={},
  feat={ms,sp,dp},
  desc={},
  qstn={},
  advq={}
}

\Entry{
  word={손위 남성형제},
  pred={엉아},
  feat={ms,sp,dp},
  desc={},
  qstn={},
  advq={}
}

\Entry{
  word={손위 남성형제의 아내},
  pred={},
  feat={ms,sp,dp},
  desc={},
  qstn={},
  advq={}
}

\Entry{
  word={손아래 여성형제},
  pred={아수},
  feat={ms,sp,dp},
  desc={},
  qstn={},
  advq={}
}

\Entry{
  word={손아래 여성형제의 남편},
  pred={},
  feat={ms,sp,dp},
  desc={},
  qstn={},
  advq={}
}

\Entry{
  word={손아래 남성형제},
  pred={아수},
  feat={ms,sp,dp},
  desc={},
  qstn={},
  advq={}
}

\Entry{
  word={손아래 남성형제의 아내},
  pred={},
  feat={ms,sp,dp},
  desc={},
  qstn={},
  advq={}
}

\Entry{
  word={아내의 모친},
  pred={},
  feat={ms,sp,dp},
  desc={},
  qstn={},
  advq={}
}

\Entry{
  word={아내의 부친},
  pred={},
  feat={ms,sp,dp},
  desc={},
  qstn={},
  advq={}
}

\Entry{
  word={아내의 손위 여성형제},
  pred={},
  feat={ms,sp,dp},
  desc={},
  qstn={},
  advq={}
}

\Entry{
  word={아내의 손위 여성형제의 남편},
  pred={},
  feat={ms,sp,dp},
  desc={},
  qstn={},
  advq={}
}

\Entry{
  word={아내의 손위 남성형제},
  pred={},
  feat={ms,sp,dp},
  desc={},
  qstn={},
  advq={}
}

\Entry{
  word={아내의 손위 남성형제의 아내},
  pred={},
  feat={ms,sp,dp},
  desc={},
  qstn={},
  advq={}
}

\Entry{
  word={아내의 손아래 여성형제},
  pred={},
  feat={ms,sp,dp},
  desc={},
  qstn={},
  advq={}
}

\Entry{
  word={아내의 손아래 여성형제의 남편},
  pred={},
  feat={ms,sp,dp},
  desc={},
  qstn={},
  advq={}
}

\Entry{
  word={아내의 손아래 남성형제},
  pred={},
  feat={ms,sp,dp},
  desc={},
  qstn={},
  advq={}
}

\Entry{
  word={아내의 손아래 남성형제의 아내},
  pred={},
  feat={ms,sp,dp},
  desc={},
  qstn={},
  advq={}
}

\Entry{
  word={남편의 모친},
  pred={},
  feat={ms,sp,dp},
  desc={},
  qstn={},
  advq={}
}

\Entry{
  word={남편의 부친},
  pred={},
  feat={ms,sp,dp},
  desc={},
  qstn={},
  advq={}
}

\Entry{
  word={남편의 손위 여성형제},
  pred={},
  feat={ms,sp,dp},
  desc={},
  qstn={},
  advq={}
}

\Entry{
  word={남편의 손위 여성형제의 남편},
  pred={},
  feat={ms,sp,dp},
  desc={},
  qstn={},
  advq={}
}

\Entry{
  word={남편의 손위 남성형제},
  pred={},
  feat={ms,sp,dp},
  desc={},
  qstn={},
  advq={}
}

\Entry{
  word={남편의 손위 남성형제의 아내},
  pred={},
  feat={ms,sp,dp},
  desc={},
  qstn={},
  advq={}
}

\Entry{
  word={남편의 손아래 여성형제},
  pred={아가씨/동생이댁},
  feat={ms,sp,dp},
  desc={},
  qstn={},
  advq={}
}

\Entry{
  word={남편의 손아래 여성형제의 남편},
  pred={},
  feat={ms,sp,dp},
  desc={},
  qstn={},
  advq={}
}

\Entry{
  word={남편의 손아래 남성형제},
  pred={},
  feat={ms,sp,dp},
  desc={},
  qstn={},
  advq={}
}

\Entry{
  word={남편의 손아래 남성형제의 아내},
  pred={},
  feat={ms,sp,dp},
  desc={},
  qstn={},
  advq={}
}

\Entry{
  word={딸의 남편},
  pred={사위},
  feat={ms,sp,dp},
  desc={V2 /ㅟ/ (사위) \\ 사회... > 사위},
  qstn={},
  advq={}
}

\Entry{
  word={아들의 아내},
  pred={메느리},
  feat={ms,sp,dp},
  desc={},
  qstn={},
  advq={}
}

\Entry{
  word={아재},
  pred={},
  feat={ms,sp,dp},
  desc={/아재/로 지칭되는 관계의 존재 여부 및 범위 확인},
  qstn={},
  advq={}
}

\chapter{문법편}
\subsection{일러두기}
일러둘 내용을 적습니다.

\section{구술발화}
\subsection{일러두기}
작성중

\subsection{음식 비교 및 선호도 (B. 가정과 생활)}
작성중

\subsection{취미 활동 묘사 (C. 자연과 일상)}
작성중

\subsection{여행 경험 및 계획 (E. 사회적 지식)}
작성중

\section{조사}
※ 질문에서 조사 노출을 최소화하기 바랍니다.

\subsection{격조사}
작성중

\subsection{보조사}
작성중

\subsection{문장 뒤 조사}
작성중

\section{종결어미}
작성중 - 표 매크로 필요

\section{연결어미}
작성중

\section{시제}
작성중

\section{관형형}
작성중

\section{부정}
작성중

\section{사동표현과 피동표현}
작성중

\subsection{사동표현}
작성중

\subsection{피동표현}
작성중

\subsection{보조용언}

\chapter{어휘 의미지도}
\section{농사와 식물}
주제별로 의미지도를 그립니다.

\section{가정과 생활}
주제별로 의미지도를 그립니다.

\section{자연과 일상}
주제별로 의미지도를 그립니다.

\section{바다}
주제별로 의미지도를 그립니다.

\section{사회적 지식}
주제별로 의미지도를 그립니다.

\section{신체}
주제별로 의미지도를 그립니다.

\section{기타 동사}
주제별로 의미지도를 그립니다.

\section{친족어}
주제별로 의미지도를 그립니다.

\chapter{사진 자료}
참고할 사진을 넣습니다.


\chapter*{부록}
\begin{appendices}

\section{어휘 체크리스트}
어휘 체크리스트를 적습니다.

\section{문법 체크리스트}
문법 체크리스트를 적습니다.

\end{appendices}


\backmatter
\makebackcover

\end{document}